\chapter{Wigilia Paschalna}

\section{Przygotowanie do obrzędów}

\subsection{Zakrystia}
\begin{itemize}
	\item Na czuwanie:
	      \begin{itemize}
		      \item alba simplex
		      \item kapa i stuła {\color{violet} fioletowa}
		      \item biret
	      \end{itemize}
	\item Na Mszę
	      \begin{itemize}
		      \item alba koronkowa
		      \item ornat, stuła, manipularz białe
		      \item komże koronkowe
	      \end{itemize}
	\item dla ks. Ireneusza na obrzędzy przygotowawcze Chrztu
	      \begin{itemize}
		      \item kapa i stuła {\color{violet} fioletowa}
		      \item rytuał wrocławski
	      \end{itemize}
\end{itemize}

\subsection{Zakrystia (do wzięcia ze sobą)}
\begin{itemize}
	\item  5 ziaren kadzidła (gruszek) --  bierze \aa1
	\item małe kropidło napełnione wodą święconą -- bierze \aa2
	\item pusty trybularz -- bierze \tt
	\item OHS --  bierze \cc2
	\item krzyż procesyjny odkryty -- bierze \ding{63}
\end{itemize}

\subsection{Kredencja przy ogniu}
\begin{itemize}
	\item biała dalmatyka, stuła, manipularz
	\item świeczka do odpalenia Paschału
	\item węgielki i sprzęt potrzebny do ich podpalenia w ogniu i wyjęcia
\end{itemize}

\subsection{Baptysterium}
\begin{itemize}
	\item chrzcielnica umyta, napełniona świeżą, czystą wodą, przyozdobiona
	      białym materiałem, zielenią i kwiatami \footnote{zgodnie z polskim zwyczajem
		      można obsypać kwiatami posadzkę w baptysterium}
	\item dodatkowy stojak na Paschał
	\item księga poświęcenia wody (np. OHS)
	\item pusty kociołek na wodę i kropidło
	\item rytuał lub inną księgę zawierającą obrzędy chrztu
	\item naczynie z solą (błogosławioną lub do pobłogosławienia)
	\item naczynia z olejem katechumenów i Krzyżmem do namaszczenia katechumena
	\item naczynie lub muszlę do udzielania chrztu
	\item czysty ręcznik do otarcia głowy ochrzczonego
	\item biała szata oraz świeca chrzcielna dla każdego katechumena
	      (przynoszone przez rodziców chrzestnych)
\end{itemize}

\subsection{Stolik w baptysterium}
\begin{itemize}
	\item oleje
	\item biała kapa i stuła
	\item garnek na wodę święconą (dla wiernych)
\end{itemize}

\subsection{Kaplica bierzmowania (św. Iwo)}
\begin{itemize}
	\item ołtarz nakryty obrusem, na predelli krzyż i  4. lichtarze ze świecami
	\item siedzenie bez oparcia dla celebransa - na środku suppedaneum, przy
	      ołtarzu;
	\item rytuał, pontyfikał lub inną księgę z obrzędami bierzmowania;
	\item krzyżmo w naczyniu z watą;
	\item tacę z watą lub odpowiednie opaski płócienne do przewiązania w miejscu
	      namaszczenia dla bierzmowanych;
	\item tacę z chlebem i solą oraz naczynie z wodą i misę do obmycia rąk
	      celebransa;
	\item gremiał płócienny dla ochrony szat przed zabrudzeniem olejami (może
	      być np. stary humerał)
\end{itemize}

\subsection{Kredensja ołtarzowa}
\begin{itemize}
	\item Ewangeliarz
	\item kielich, ampułki, cyborium z komunikantami
	\item akolitki
	\item dwie świeczki woskowe dla akolitów (podczas \textit{Exultetu})
	\item relikwiarze
	\item kwiaty (za ołtarzem)
	\item wszystkie dzwonki jakie mamy
	\item poduszki fioletowe
	\item mała czerwona stuła na krzyż
	\item mszaliki z łacińskim tekstem dla kontrolowania przebiegu liturgii
	\item 6 świec ołtarzowych 
\end{itemize}

\subsection{Ołtarz główny}
\begin{itemize}
	\item antypendium białe, na nim założony {\color{violet}fioletowy} materiał,
	      który można łatwo usunąć.
	\item 6 białych świec na złotych świecznikach
	\item odkryty krzyż złoty
	\item mszał otwarty na proroctwach
	\item obrus rozwinięty
	\item koronka ołtarzowa zawinięta
\end{itemize}

\subsection{Inne}
\begin{itemize}
	\item w pierwszym rzędzie ławek: klęcznik lub ławka z zarezerwowanym
	      miejscem dla katechumena i rodziców chrzestnych
	\item mały klęcznik okryty na biało dla katechumena do przyjęcia Pierwszej
	      Komunii Świętej
\end{itemize}

