\section{Liturgia światła}
	
	\begin{itemize}
		\item Do poświęcenia ognia wychodzimy nastąpująco:
		\begin{itemize}
			\item \tt~ z pustą kadzielnicą, 
			\item \ding{63},
			\item \aa1 z kropidłem, 
			\item \aa2 z tacką z granami i rylcem,
			\item \mm1 z paschałem, 
			\item \cc2 trzymając w ręku dalmatykę i stułę, 
			\item \cc1 z kropidłem i OHS, 
			\item \ii~ w kapie.
		\end{itemize}
		\item turyfer po dojściu do ogniska od razu wrzuca do środka węgielki, aby się rozpaliły.
		\item stajemy przed kościołem w następujących pozycjach:
		\item (tutaj będzie obrazek)
		\item następuje oświęcenie ognia -- modlitwa
		\item potem pokropienie ogniska -- \aa1 podaje kropidło bezpośrednio \ii~ i potem je odbiera Także i on przytrzymuje kapę, aby się \textbf{nie spaliła}.
		\item po poświęceniu ogniska \tt~ nakłada do trybularza węgiel, następnie zasypanie i okadzenie. Teraz turyfer trzyma kapę, aby się \textbf{nie spaliła}.
		\item jeden z kantorów odpala od ogniska świeczkę i idzie z nią do kościoła (może to być M. Rumin)
		\item następnie naprzeciwko \ii~ staje \mm z paschałem. Po jego lewej staje \aa2 z
		tacką z rylcem. \ii~ kreśli wszystkie litery i cyfry.
		\item poświęcenie gran: trzykrotne pokropienie i okadzenie (\aa1 I podaje i odbiera kropidło, \tt~ podaje trybularz: bez zasypania!)
		\item włożenie wszystkich gran w paschał.
		\item \cc1 odpala świeczkę od ognia i podaje \ii. Ten odpala paschał i potem go błogosławi.
		\item \ii~ przebiera się w białą stułę i dalmatykę. Pomaga mu w tym \cc2 oraz \aa2.
		\item dopiero teraz nastepuje zasypanie.
		\item procesja z paschałem do kościoła: \tt, \ding{63}, \ii~ z paschałem, \cc,
		\aa i \mm1, który zabiera tackę i kropidło.
		\item trzykrotne Lumen Christi: zaraz po wejściu do kościoła na stopniu koło figury św. Jadwigi,
		potem w połowie nawy głównej (tam gdzie siedzą wierni) i w prezbiterium.
		\item UWAGA! Tu zmiana w stosunku do lat poprzednich. Zawsze wchodziliśmy do kościoła i skręcaliśmy w prawo. Tym razem w lewo. Uznałem, że paschał ma się kierować bezpośrednio do prezbiterium, a nie być obnoszony w procesji dookoła kościoła.
	\end{itemize}
	
	