\section{Liturgia chrzcielna}

\begin{itemize}
	\item po ostatnim proroctwie:
	      \begin{itemize}
		      \item \aa1 odpala od paschału akolitki
		      \item \ii wraz z \cc klęka po stronie lekcji na pierwszym stopniu
		      \item \ding{63} idzie po krzyż procesyjny
	      \end{itemize}
	\item gdy \ii uklęknie kantorzy zaczynają śpiew \textit{Litanii do Wszystkich Świętych}
	\item w tym czasie ustawiamy procesję do chrzcielnicy:
	      \begin{itemize}
		      \item \mm1 z \textbf{zapalonym} paschałem
		      \item \ding{63} w towarzystwie \aa
		      \item pozostali ministranci i kantorzy
	      \end{itemize}
	\item procesja rusza po słowach \textit{Sancta Trinitas Unus Deus: miserere nobis}
	\item (tutaj będzie rysunek)
	\item po dojściu do chrzcielnicy kantorzy śpiewają kantyk \textit{Sicut
		      Cervus} i dopiero po nim zaczyna się poświęcenie wody chrzcielnej
	\item stoimy w następującym porządku:
	\item (tutaj będzie obrazek)
	\item następuje modlitwa \textit{Omnipotens sempiterne Deus, respice}
	\item następnie druga modlitwa i poświęcenie wody (jak zwykle)
	\item potem zasypanie i okadzenie wody chrzcielnej
	\item zamiana kapy fioletowej
	\item odnowienie przyrzeczeń chrzcielnych (przy chrzcielnicy) i pokropienie
	\item na pokropienie chór śpiewa stosowną pieśń (np.\textit{Przez chrztu
		      świętego wielki dar})
	\item oleje i kapa biała leżą przygotowane na stoliku, podobnie kociołek na
	      wodę święconą oraz garnek na nią, aby wierni mogli sobie jej więcej
	      nabrać do domu
	\item \cc1 podaje \ii~ oleje
	\item \cc2 po nabraniu wody do kociołka i do garnka ucieka z pola widzenia
	      bokiem i zaczyna przygotowywać prezbiterium do Mszy (jeszcze jak
	      reszta jest w chrzcielnicy). Trzeba odstawić stojak na paschał na
	      stronę ewangelii, odstawić pulpit, powiesić stułę na krzyż, ustawić
	      relikwie, kwiaty. Na razie nie rozkładać obrusów!!!
	\item po okadzeniu wody chrzcielnej \tt~ też od razu ucieka bokiem i idzie
	      pomagać \cc2
	\item po skończonych odnowieniach ustawiamy procesję z powrotem do Ołtarza w
	      tym samym porządku co poprzednio
	\item kantorzy intonują drugą część \textit{Litanii do Wszystkich Świętych}
	\item po przybyciu do Ołtarza \mm1 odstawia Paschał na swoje miejsce
	\item \cc1 zapala od paschału świece na Ołtarzu i rozkłada obrusy -- pomaga
	      mu \mm1
	\item \cc2 zamienia się z \cc1 miejscami i klęka z \ii~ po stronie lekcji na
	      stopniu
	\item \aa~ wraz z \ding{63} stoją przy kredensie
	\item kantorzy klękają na stopniach prezbiterium i kontynuują litanię
	\item \tt~ idzie już bokiem do zakrystii i przebiera się
	\item po wezwaniu \textit{Peccatores} \ii~ wstaje i wraz ze wszystkimi udaje
	      się do zakrystii, gdzie zmieniamy komże na ładniejsze, \ii~ ubiera
	      ornat biały.
\end{itemize}
