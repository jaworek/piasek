\section{Msza Św. i procesja do Bożego Grobu}

	\subsection{Msza Św.}

	\begin{itemize}
		\item po przebraniu się od udajemy się procesyjnie do Ołtarza i rozpoczyna się Msza św. 
		\item nie ma modlitw u stopni Ołtarza, nie ma Confiteor, Od razu Aufer a nobis 
		\item zasypanie i okadzenie
		\item \aa1 niech ma	na uwadze, że nie musi Mszału zabierać z Ołtarza, bo ten jest na stoliku; ale trzeba go po okadzeniu ustawić na ołtarz
		\item \cc2 posługuje przy Ołtarzu; teraz on jest ważniejszy
		\item \aa~ na \textit{Gloria} dzwonią dzwonkami; w tym czasie w sygnaturkę uderza \tt~ i \cc1
		\item Msza biegnie normalnie; \cc1 idzie w tym czasie do chrzcielnicy i sprząta co nieco.
		\item przed komunią (po przyjęciu komunii przez \ii) \cc2 zaczyna \textit{Confiteor}!!!
		\item po komunii są \textit{Laudesy}. Pamiętamy, że na słowa \textit{Benedictus} żegnamy się znakiem krzyża
		\item potem	jest zasypanie i okadzenie ołtarza, \ii, usługujących i wiernych (jak w trkacie ofiarowania)
		\item po \textit{Ite missa est, alleluia, alleluia} i błogosławieństwie nie ma ostatniej ewangelii. 
		\item \ii~ przebiera ornat na kapę
		\item potem procesyjnie udajemy się do Grobu. 
		\item \cc2 zabiera ze sobą dzwonki
		\item \cc1 przygotuje Ołtarz i za chwilę dojdzie do reszty.
	\end{itemize}

	\subsection{Przy Grobie}
	
	\begin{itemize}
		\item ustawiamy się następująco:
		\item (tutaj będzie rysunek)
		\item wszyscy klękamy
		\item \ii intonuje \textit{Gloria, tibi Trinitas}
		\item po skończeniu modlitw \mm1 wstaje i bierze figurę Zmartwychwstałego i staje obok \ding{63}
		\item procesja rusza jak tylko rozpocznie się śpiew pieśni po polsku i wygląda następująco:
		\item (tutaj obrazek)
		\item po przybyciu do Ołtarza, \aa~ i \ding{64} stają przy kredensie, \mm1 ustawia figurę zmartwychwstałego po stronie ewangelii i schodzi na swoje miejsce, \tt~ staje po swojej stronie, \cc2 staje u stopni Ołtarza po prawej stronie \ii, a \cc1 jak już odstawi parasolkę to staje po lewej stronie \ii
		\item śpiewamy na stojąco \textit{Te Deum}
		\item potem jest błogosławieństwo (nie ma podnoszenia krzyża, nie ma
		\textit{Przed tak wielkim sakramentem})
		\item na koniec \ii intonuje \textit{Regina Caeli}, odmawia modlitwę i po
		tym schodzimy procesyjnie do zakrystii
	\end{itemize}

	\bigskip

	\begin{center}
		\Large
		\textbf{Wesołych Świąt Zmartwychwstania Pańskiego!}
	\end{center}
	










	