\section{Msza Święta - ważniejsze zmiany}
\begin{itemize}
	\item wszyscy odkładają palmy
	\item nie odmawia się modlitw u stopni
	\item bezpośrednio po przyklęknięciu następuje ucałowanie ołtarza i
	      nałożenie kadzidła
	\item po odmówieniu \textit{Introitu} i \textit{Kyrie} $\mathcal{I}$ i
	      $\mathcal{D}$ oraz $\mathcal{S}$ krótką drogą schodzą do sedilii
	\item podczas lekcji na słowa \textit{in nomine Iesu omne genu flectatur}
	      wszyscy przyklękają
	\item procesja przed Ewanglią, jak i funkcje tworzących ich ministrantów
	      pozostają (prawie) bez zmian względem Mszy śpiewanej (za wyjątkiem
	      nietrzymania w jej czasie paramentów litrugicznych); więcej patrz
	      \hyperref[E]{Dodatek}
	\item chór podczas Ewangelii wykonuje skłony w kierunku krzyża ołtarzowego,
	      a śpiewacy ``przed siebie``
	\item po przyjęciu przez $\mathcal{I}$ Komunii Św. i wyciągnięciu puszki z
	      tabernakulum $\mathcal{D}$ \textbf{sam} odmawia \textit{Confiteor}, po
	      którym nastepuje \textit{Misereatur} -- \textbf{wszyscy} są w tym
	      czasie pochyleni. Prostujemy się dopiero na \textit{Indulgenciam}
	\item nie ma ostatniej Ewangelii

\end{itemize}