\section{Msza Święta - ważniejsze zmiany}
\begin{itemize}
	\item wszyscy odkładają palmy
	\item nie odmawia się modlitw u stopni
	\item bezpośrednio po przyklęknięciu następuje ucałowanie ołtarza i
	      nałożenie kadzidła
	\item po odmówieniu \textit{Introitu} i \textit{Kyrie} \ii~ i \dd~ oraz \ss~
	      krótką drogą schodzą do sedilii
	\item \ii~ (cały czas w asyście \dd) po przeczytaniu lekcji (podczas której
	      nie przyklęka) czyta dalsze propria \footnote{\textit{Graduał} i
		      \textit{Traktus}}. Przyklęka ze wszystkimi, wtedy gdy \ss~ zaśpiewa
	      \textit{\dots in nomine Iesu omne genu flectatur \dots}. Jeżeli do
	      czasu błogosławieństwa \ss~ nie przeczyta wszystkiego, robi przerwę na
	      pobłogosławienie. Siada dopiero po przeczytaniu wszystkiego do
	      Ewangelii wyłącznie lub po błogosławieństwie \ss.
	\item chór śpiewa pełną wersję \textit{Graduału} oraz  \textit{Traktusa}
	\item procesja przed Ewangelią, jak i funkcje tworzących ich ministrantów
	      pozostają (prawie) bez zmian względem Mszy śpiewanej (za wyjątkiem
	      nietrzymania w jej czasie paramentów litrugicznych); więcej patrz
	      \textit{\nameref{sec:dodatek_palm}}
	\item chór podczas Ewangelii:

	      \begin{itemize}
		      \item trzyma w rękach palmy
		      \item wykonuje skłony w kierunku krzyża ołtarzowego
		      \item śpiewacy wykonują skłony przed siebie
		      \item klękamy tak jak stoimy przed siebie
		  \end{itemize}
		  
	\item po przyjęciu przez \ii~ Komunii Św. i wyciągnięciu puszki z
	      tabernakulum \dd~ \textbf{sam} odmawia \textit{Confiteor}, po którym
	      nastepuje \textit{Misereatur} -- \textbf{wszyscy} są w tym czasie
	      pochyleni. Prostujemy się dopiero na \textit{Indulgenciam}
	\item nie ma ostatniej Ewangelii
	\item antyfonę \textit{Ave Regina Caelorum} zastępuje się sekwencją
	      \textit{Stabat Mater} na melodię Watykańską.

\end{itemize}