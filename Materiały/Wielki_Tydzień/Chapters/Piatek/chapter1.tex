\chapter{Wielki Piątek}

\section{Przygotowanie do obrzędów}

\subsection{Ołtarz główny}

\begin{itemize}
    \item zupełnie ogołocony
    \item tabernakulum puste, obok kluczyk
    \item na tabernakulum podstawa pod krzyż
    \item na najwyższym stopniu jedna fioletowa poduszka do prostracji
    \item za ołtarzem odkryty krzyż procesyjny
\end{itemize}

\subsection{Kredencja}

\begin{itemize}
    \item przykryta normalnie obrusem
    \item jeden obrus ołtarzowy
    \item pulpit pod mszał, razem z mszałem
    \item księga OHS
    \item pozostałe dwie fioletowe poduszki pod krzyż
    \item ręczniczek do wycierania krzyża
    \item fioletowa bursa z korporałem
    \item dwie pateny komunijne
    \item dwie świece sanctusowe
    \item vasculum i ręczniczek
    \item akolitki
    \item monstrancja
    \item welon do monstrancji
\end{itemize}

\subsection{Zakrystia/kaplica}

\begin{itemize}
    \item czarna kapa
    \item fioletowy ornat ze stułą
    \item fioletowa kapa
    \item zasłonięty krzyż do adoracji
    \item 4 świeczniki z żółtymi świecami (i zapalniczka)
    \item biały welon naramienny
    \item dwa trybularze, jedna łódka
    \item ombrelino
\end{itemize}

\subsection{Inne}

\begin{itemize}
    \item przy sedilli dodatkowe dwa miejsca dla ceremoniarzy (za akolitami)
    \item wysoki pulpit do śpiewu lekcji
    \item teksty Pasji do śpiewania
    \item dwie kołatki
    \item schodki
    \item fioletowa stuła dla ks. Ireneusza do Komunii
    \item świece na procesję dla ministrantów (z okapnikami, zapalniczki)
\end{itemize}
