	\section{Wystawienie Najświętszego Sakramentu}
	
		\begin{center}
			Potrzebne: kapa, monstrancja, welon naramienny
		\end{center}
	
		\textbf{Zasady ogólne:}
		\begin{itemize}
			\item Ministranci posługujący nie klękają podczas wystawienia na stopniu ołtarza, ale na podłodze!
			\item Podczas wystawienia na ołtarzu nie powinny stać tablice ołtarzowe, relikwie, kielich, mszał
			\item Jeśli jest piąty ministrant, który słucha Mszy w chórze, asystuje on kapłanowi razem z ceremoniarzem podczas wystawienia. Przynosi on kapę do przebrania celebransowi, a także podtrzymuje brzegi kapy podczas błogosławieństwa.
			\item Jeśli nie ma piątego ministranta, kapę do przebrania przynosi celebransowi turyfer, a podczas błogosławieństwa brzeg kapy podtrzymuje jeden z akolitów.
			\item Akolici w odpowiednich momentach podają ceremoniarzowi, a także odbierają od niego: książki z modlitwami, welon naramienny, inne potrzebne rzeczy.
			\item Podczas wystawienia nie błogosławi się kadzidła, nie całuje się dłoni celebransa, nie oddaje się nikomu rewerencji (pokłonów). W chórze nie siadamy, lecz pozostajemy w pozycji klęczącej.	
		\end{itemize}
	
		\textbf{Krok po kroku:}
		\begin{itemize}
			\item Przed Mszą wyznacza się ministranta z chóru do pełnienia funkcji asystenta przy wystawieniu. Jeśli jest tylko 4 ministrantów, kapę celebransowi przynosi turyfer, a podczas błogosławieństwa brzegi kapy podtrzymuje jeden z akolitów.
			\item Asystent (jeśli go nie ma, to turyfer) – podczas modlitwy Postcommunio udaje się do bocznego ołtarza po kapę.
			\item Po odśpiewaniu \textit{Benedicamus Domino} kapłan z ceremoniarzem klęka przed ołtarzem i udaje się do sedilli.
			\item Asystent (lub turyfer) trzyma kapę jak parawan przy sedilli, w tym czasie ceremoniarz pomaga kapłanowi zdjąć ornat i manipularz. Na znak ceremoniarza asystent zakłada celebransowi kapę.
			\item Turyfer udaje się po kadzielnicę i przynosi ją.
			\item W tym czasie akolici :
			\begin{itemize}
				\item kładą tablice ołtarzowe tak, aby nie przeszkadzały w wystawieniu
				\item jeśli kapłan tego nie zrobił, rozkładają korporał z bursy obok tabernakulum
				\item ściągają z ołtarza relikwie
				\item ściągają z ołtarza mikrofon i umieszczają go na drugim stopniu ołtarza po stronie Ewangelii, bliżej środka, wraz ze statywem
				\item umieszczają na ołtarzu monstrancję po stronie Ewangelii, otworem zwróconą w kierunku korporału
				\item wracają na swoje miejsca
			\end{itemize}
			\item Kiedy akolici wypełnili swoje zadania, ceremoniarz daje znak celebransowi.
			\item Celebrans w asyście ceremoniarza i asystenta (jeśli jest) udają się do ołtarza i przyklękają. Asystent zajmuje miejsce po stronie Ewangelii, a ceremoniarz po stronie Epistoły. Celebrans wchodzi po stopniach ołtarza do tabernakulum i dokonuje wystawienia.
			\item Kiedy kapłan wszedł na ostatni stopień, ceremoniarz daje znak wszystkim do uklęknięcia na dwa kolana.
			\textbf{Ceremoniarz i asystent klękają na podłodze, nie na stopniu ołtarza!}
			\item  Kiedy kapłan dokona wystawienia i schodzi na dół, klęka na pierwszym stopniu ołtarza. Razem z ceremoniarzem i asystentem wykonują głęboki ukłon przed Sanctissimum.
			\item Potem wstają, turyfer podchodzi do ceremoniarza. Następuje zasypanie kadzidła.Asystent podtrzymuje prawy brzeg kapy, ceremoniarz podaje łódkę. Nie prosi się kapłana o błogosławieństwo, nie całuje się łyżeczki.
			\item  Po zasypaniu ceremoniarz oddaje łódkę turyferowi i odbiera od niego trybularz.
			\item Po uklęknięciu ceremoniarz podaje celebransowi trybularz bez pocałunków.
			\item Po wykonanym głębokim ukłonie okadza się Najświętszy Sakrament. Ceremoniarz i asystent podtrzymują brzegi kapy.
			\item Ceremoniarz klęcząc oddaje trybularz turyferowi, który odchodzi na bok.
			\item Akolita podaje klęczącemu ceremoniarzowi książkę z modlitwami. Ceremoniarz podaje ją kapłanowi. Asystent umieszcza mikrofon przed kapłanem.
			\item Brzegi kapy układa się elegancko na drugim stopniu ołtarza, tak żeby zakrywały przestrzeń przed kapłanem – Nie powinny niechlujnie zwisać po bokach!
			\item W razie potrzeby ceremoniarz i asystent pomagają celebransowi z książką i modlitwami.
			\item Po skończonych modlitwach, kiedy intonuje się \textit{Tantum ergo}, na słowa \textit{Veneremur cernui} (\textit{Upadajmy wszyscy wraz}) wszyscy skłaniają się nisko.
			\item  Potem turyfer zbliża się do ceremoniarza i następują drugie zasypanie i okadzenie Sanctissimum.
			\item Po okadzeniu akolita podaje ceremoniarzowi welon naramienny.
			\item Kapłan sam wstaje i śpiewa oracje przed błogosławieństwem. (Jeśli potrzeba, podaje się mu tekst)
			\item  Po oracjach kapłan klęka. Celebrans podaje mu welon naramienny.
			\item Celebrans, ceremoniarz i asystent (lub jeden z akolitów) wstępują na stopnie ołtarza. Podczas błogosławieństwa Sanctissimum ceremoniarz i asystent klęcząc na najwyższym stopniu ołtarza podtrzymują brzegi kapy. Po błogosławieństwie przyklękają na najwyższym stopniu razem z celebransem.
			\item Akolita dzwoni dzwonkiem podczas błogosławieństwa.
			\item Po przyklęknięciu na najwyższym stopniu celebrans ponownie klęka na pierwszym stopniu ołtarza, a ceremoniarz i asystent klękają na podłodze. Wykonują głęboki ukłon.
			\item Ceremoniarz odbiera welon od celebransa. Jeden z akolitów podchodzi do ceremoniarza, odbiera od niego welon, składa go i kładzie na kredencji.
			\item Jeśli odmawia się \textit{Divinae Laudationes} (\textit{Niech będzie Bóg uwielbiony...}), ceremoniarz podaje tekst celebransowi.
			\item Celebrans ponownie wstępuje sam na najwyższy stopień i dokonuje schowania Najświętszego Sakramentu.
			\item Akolita dzwoni dzwonkiem na zamknięcie tabernakulum i wtedy ceremoniarz daje wszystkim znak do powstania.
			\item Wyjście z prezbiterium jak zawsze po każdej Mszy Świętej.
		\end{itemize}