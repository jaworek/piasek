%\chapter{}

\section{Sposoby służenia na Piasku}

\subsection{Rodzaje Mszy Św.}

\begin{itemize}
	\item \textbf{recytowana} -- z jednym lub dwoma ministrantami
	\item \textbf{recytowana świąteczna} (także kolędowa) -- z jednym lub dwoma,
	      w niedziele i święta poza wielkim postem, klęczy się na niej i stoi
	      jak na mszy śpiewanej
	\item \textbf{śpiewana \textit{na dwóch}} -- msza śpiewana bez \cc~ i \tt~
	\item \textbf{śpiewana z okadzeniem} -- standardowa msza niedzielna z \cc~ i
	      \tt~
	\item \textbf{ śpiewana \textit{bardziej uroczysta}} -- bardziej uroczysta
	      msza świąteczna	(np. Wigilia Zesłania Ducha Św.), z lucenarium,
	      obrusem komunijnym, okadzeniem i ew. dodatkowymi obrzędami
	\item \textbf{solenna} -- z asystą \dd~ i \ss, lucenarium
\end{itemize}

\noindent Przy zastosowaniu bardziej uroczystych form liturgicznych -- jeśli
jest duża ilość ministrantów -- wyznacza się drugiego ceremoniarza, który
\begin{itemize}
	\item prowadzi procesję wejścia i inne procesje, które także ustawia przed
	      wyruszeniem. Dba o dobre tempo
	\item wprowadza duchownych i ministrantów do chóru
	\item w razie potrzeby zajmuje miejsce przy \cc1, aby wspólnie asystować
	      \ii.
	\item kontroluje na bieżąco, czy czegoś nie brakuje, czy nie powinien czegoś
	      donieść itp.
	\item wprowadza lucenarium i procesję „dwójkami” do przyjmowania Komunii Św.
\end{itemize}
