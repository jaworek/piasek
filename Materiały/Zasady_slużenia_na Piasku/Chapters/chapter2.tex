\section{Aspersja}
	
		\begin{center}
			Potrzebne: kociołek z wodą, kropidło, książka z modlitwami, \\ornat i manipularz na sedilii.
			\ii~ ubrany w kapę w kolorze dnia.
		\end{center}
	
		\textbf{Zasady ogólne:}
		\begin{itemize}
			\item na wejście nie niesie się kadzielnicy
			\item na początku aspersji klękają tylko \ii~ i \cc~
			\item ministrantów kropi się sprzed stopni ołtarza jednym bądź dwoma rzutami
			\item \aa\aa~ odbierają od ceremoniarza użyte już przedmioty
			\item \aa\aa~ asystują \ii~ podczas przebierania się w ornat
		\end{itemize}
	
		\bigskip
	
		\textbf{Krok po kroku:}
		\begin{itemize}
			\item Procesja wchodzi do prezbiterium w zwykłej kolejności. \tt~ nie niesie			kadzielnicy. \aa\aa~ stawiają świece na kredencji.
			\item \ii~ poza mszami solennymi asystuje sam \cc.
			\item W prezbiterium \cc~ odbiera od \ii~ biret, odnosi go na sedillę.
			\item Kiedy \ii~ dotrze do ołtarza, przyklęka razem z \cc.
			\item \ii~ klęka na dwa kolana na pierwszym stopniu ołtarza, a \cc~ na podłodze z jego prawej strony. Pozostali ministranci nie klękają, lecz stoją.
			\item \cc~ z zachowaniem pocałunków podaje \ii~ kropidło.
			\item \ii~ intonuje \textit{Asperges me} lub \textit{Vidi aquam}. Jednocześnie kropi stopnie ołtarza wodą święconą.
			\item \ii~ kreśli kropidłem na swoim czole znak krzyża, potem podaje kropidło do przeżegnania się \dd~ i \ss, jeśli ich nie ma - \cc. Potem wstają. \cc~ podnosi kociołek z wodą.
			\item \ii~ stojąc przed stopniami ołtarza kropi jednym rzutem najpierw \aa\aa, potem jednym rzutem ministrantów w chórze.
			\item \ii~ i \cc~ przyklękają przed ołtarzem, po czym idą przez kościół kropiąc lud wodą święconą. Zaczynają od strony epistoły. \cc~ podaje	\ii~ wodę z kociołka i lekko podtrzymuje kapę, żeby \ii~ mógł swobodnie kropić.
			\item Podczas wersu \textit{Gloria Patri et Filio et Spiritui Sancto} \ii~ zwraca się w kierunku ołtarza i wspólnie z \cc~ wykonują głęboki ukłon.
			\item Na \textit{Sicut erat in principio...} idą kropić dalej. Kiedy zmieniają kierunek i wracają w kierunku ołtarza, przyklękają pośrodku.
			\item Po \textit{Gloria Patri} \aa\aa~ przechodzą ze swojego miejsca przy kredencji na miejsca przy sedilii. \tt~ podsyca ogień i przygotowuje się do wejścia z nim do prezbiterium.
			\item Kiedy \ii~ i \cc~ wróca do prezbiterium, \aa~ stojący bliżej ołtarza odbiera od \cc~ kociołek i kropidło i odnosi je na kredencję.
			\item \cc~ podaje \ii~ książkę z modlitwami i wskazuje mu właściwe modlitwy.
			\item Po odśpiewaniu modlitw \cc~ odbiera książkę.
			\item \ii~ i \cc~ przyklękają i udają się do sedilii.
			\item \aa~ bliżej ołtarza odbiera od ceremoniarza książkę i zanosi ją na kredencję.
			\item \aa~ stojący bliżej nawy ściąga z \ii~ kapę i trzyma ją jak parawan, zasłaniając zmieniającego szaty \ii~.
			\item \cc~ podaje \ii~ manipularz, pomaga mu ubrać ornat i daje znak \aa~ trzymającemu kapę, że ten może już odejść.
			\item \ii~ z \cc~ udają się do ołtarza i rozpoczynają Mszę od modlitw u stopni ołtarza.
		\end{itemize}