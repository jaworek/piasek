\section{Szczegóły dotyczące głównie mszy bardziej uroczystych i solennych}

		Na tych mszach często występują dodatkowe elementy, niezbyt często praktykowane, dlatego warto będzie raz na zawsze ustalić odpowiednią wersję ich wykonania, we wszystkich zdarzających się wariantach.\\
		
		\noindent W miarę możliwości do dużych mszy wyznaczamy dwóch ceremoniarzy 1Crm zwyczajnie asystuje kapłanowi przy ołtarzu 2Crm prowadzi procesje i lucenarium, ustawia dwójki do komunii,a także na bieżąco uzupełnia braki i niedociągnięcia.\\
		
		Kolejność okadzania (po celebransie)
		\begin{itemize}
			\item w solennej 
				\begin{itemize}
					\item diakon okadza 
						\begin{itemize}
							\item księży w chórze
							\item kleryków w chórze
							\item subdiakona
						\end{itemize}
					\item turyfer okadza
						\begin{itemize}
							\item diakona
							\item ceremoniarza
							\item akolitów
							\item ministrantów
							\item lud
						\end{itemize}
				\end{itemize}
			\item w śpiewanej turyfer okadza
				\begin{itemize}
					\item księży w chórze (każdego osobno)
					\item kleryków (w chórze lub służących do Mszy)
					\item ceremoniarza (jeśli nie jest klerykiem)
					\item akolitów
					\item ministrantów
					\item lud
				\end{itemize}
		\end{itemize}
		
		