\section{Liturgia słowa}

\begin{itemize}
	\item z zakrystii wychodzi procesja krótką drogą, \aa\aa~ nie niosą
	      akolitek, a \tt~ nie niesie trybularza
	\item po przybyciu do prezbiterium wszyscy przyklękamy przy stopniach
	      ołtarza i rozchodzimy się na swoje miejsca:
	      \begin{itemize}
		      \item \ii~ idzie po stopniach do ołtarza, całuje ołtarz i udaje
		            się do mszału
		      \item \cc1 pomaga \ii~ wejść, a potem idzie do kredencji
		            %   \item \tt~ idzie do kredencji
		      \item \cc3 idzie do chóru
		      \item \cc2 idzie do kredencji
		      \item \aa1 i \aa2 idą do kredencji
		      \item reszta idzie do chóru
	      \end{itemize}
	\item \ii~ czyta po cichu tekst proroctwa (ewentualnie także kantyku)
	      \footnote{Towarzyszy mu \cc1 ze zniczem (gdy jest zbyt ciemno)} i, po
	      skłonie do krzyża, krótką drogą udaje się do sedilli

	\item na znak \cc1~ \ii~ wstaje i razem, krótką drogą udają się do Mszału
	\item \ii~ śpiewa \textit{Oremus} ze skłonem\footnote{nie ma
		      \textit{Flectamus genua} oraz \textit{Levate}}, a potem śpiewa
	      orację
	      \footnote{nie ma \textit{Flectamus Genua}}
	\item po skończonej oracji czyta po cichu proroctwo i cykl się zamyka
	\item \aa\aa~ podczas czytania przez \ii~ lekcji stoją przy kredencji, a
	      kiedy \ii~ schodzi do sedilli udają się tam razem z nim
	\item chór siedzi podczas lekcji a stoi podczas oracji
	\item kantorzy śpiewający
	      \begin{itemize}
		      \item jeśli proroctwo poprzedza traktus to podchodzą do ambonki w
		            trakcie jego trwania
		      \item jeśli proroctwa nie poprzedza traktus \footnote{Kantyku nie
			            ma tylko po pierwszym proroctwie} to podchodzą do
		            ambonki pod koniec poprzedzającego proroctwa i stoją z
		            poprzednim kantorem, który odchodzi od ambonki po oracji
		            (stoją obok siebie podczas trwania oracji)
	      \end{itemize}
\end{itemize}