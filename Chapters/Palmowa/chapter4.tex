\section{Dodatek}
	  \subsection{Losy subdiakona krucyfera}
	    \begin{itemize}
	    \item subdiakon krucyfer ( \ding{63} ), ubrany w albę, cingulum i czerwoną tunicellę 
	    \item nie odbiera palmy od $\mathcal{I}$. 
	    \item po skończonej procesji idzie do kaplicy zimowej, gdzie ściąga westymenta subdiakona, zakłada komżę, bierze biret i przechodzi do chóru.
	    \end{itemize}
    
  %\newpage
  
  \subsection{Prosecja do Ewangelii - dla zaawansowanych}
  \label{E}
    \subsubsection*{\textbf{Celebrans z subdiakonem i ceremoniarzem }}
      \begin{itemize}
      \item kiedy skończy się śpiew Graduału i Traktusa, $\mathcal{I}$ wraz z $\mathcal{S}$ i $\mathcal{C}1$ podchodzą do ołtarza - przyklękają 
      \item $\mathcal{I}$ wchodzi na najwyższy stopień i odmawia modlitwę \textit{Munda cor} 
      \item \ss~ w tym czasie przenosi mszał,
      \end{itemize}
      
    \subsubsection*{\textbf{Diakon}}
      \begin{itemize}
       \item przy pomocy $\mathcal{C}2$ zdejmuje dalmatykę i manipularz 
       \item od $\mathcal{C}2$ otrzymuje księgę z tekstem Pasji, a następnie zajmuje miejsce pomiędzy dwoma śpiewakami - $\mathcal{S}1$ i $\mathcal{S}2$
      \end{itemize}
      
    \subsubsection*{\textbf{Ale jak to wygląda?}}

    \begin{enumerate}\centering
     \item[] $\mathcal{I}$
     \item[] (stopnie ołtarza)
     \item[] $\mathcal{S}$
     \item[] 
     \item[] 
     \item[] $\mathcal{S}1$~~~$\mathcal{D}$~~~$\mathcal{S}2$
     \item[] $\mathcal{C}2$
    \end{enumerate}
    
   % \newpage
    
    \begin{itemize}
     \item na znak $\mathcal{C}2$ wszyscy oprócz $\mathcal{I}$ przyklękają 
     \item $\mathcal{S}$ przechodzi ze Mszałem na stronę Ewangelii i pozostaje przy nim asystując $\mathcal{I}$ przy kartkach 
     \item $\mathcal{C}1$, $\mathcal{T}$, $\mathcal{A}1$, $\mathcal{A}2$ \textbf{nie} ustawiają się jak na Mszy śpiewanej (zostają na swoich 'bazach') i skłaniają się do krzyża ołtarzowego razem z $\mathcal{I}$ \footnote{Ceremoniał o. Małaczyńskiego, str. 65, pkt. 47(dla rytu uroczystego): „Jeżeli Męka Pańska nie jest śpiewana celebrans pod koniec traktusa odmawia „Munda cor” i czyta Mękę Pańską po stronie Ewangelii. Równocześnie należy ją odczytać wiernym  po polsku."}
     \item[]
     \item $\mathcal{C}2$ prowadzi $\mathcal{S}1$, $\mathcal{D}$, $\mathcal{S}2$ do ambony – gdzie następuje śpiew Pasji po polsku
     \item wszystkie skłony podczas Pasji wykonujemy w stronę ewangeliarza/lekcjonarza (nie w stronę Mszału),
     \item[]
     \item po odśpiewanej Pasji po polsku $\mathcal{I}$ staje przed środkiem ołtarza, a $\mathcal{S}$ za nim na swoim stopniu. 
     \item $\mathcal{D}$, $\mathcal{S}1$, $\mathcal{S}2$ i $\mathcal{C}2$ przyklękają na środku, po czym $\mathcal{C}2$ prowadzi $\mathcal{D}$ do sedilli i pomaga mu założyć dalmatykę i manipularz 
     \item $\mathcal{D}$ zajmuje swoje miejsce za $\mathcal{I}$ i przyklęka -- $\mathcal{I}$ rozpoczyna \textit{Credo}
    \end{itemize}

%  \section{Do poprawy}
%    \begin{itemize}
%     \item {\color{red}Przypomnieć ministrantom, żeby na Ewangelie skłaniali głowy \textbf{do Ewangeliarza}, a nie do ołtarza.}
%     \item \sout{Jeśliby dobrze poukładać palmy na stoliku, tak żeby całość była stabilna, to można by się pokusić o 'wjechanie' nim na środek jak już przejdzie cała procesja.}
%    \end{itemize}