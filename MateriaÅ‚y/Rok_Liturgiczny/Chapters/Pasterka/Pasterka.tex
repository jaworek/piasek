\chapter{Pasterka}

\section{Procesja wejścia}

\begin{itemize}
	\item W kościele należy wygasić światła.
	\item Kapłan ubrany w kapę koloru białego bez akompaniamentu organów
	      się do żłóbka ustawionego w prezbiterium. (W tym czasie jeden z
	      usługujących może uderzyć 12 razy w gong)
	\item Po przyjściu kładzie figurkę Dzieciątka i przykrywa ją welonem.
	      Następnie staje przy krześle wraz z asystą.
	\item Kantor staje na ambonie i śpiewa Kalendę: \smallbreak
	      %
	      \inde{Octavo Kalendas Januarii Luna N.\\
		      Innumeris transactis saeculis a creatione mundi, quando in
		      principio Deus creavit caelum et terram et hominem formavit ad
		      imaginem suam;\\
		      permultis etiam saeculis, ex quo post diluvium Altissimus in
		      nubibus arcum posuerat, signum fœderis et pacis;\\
		      a migratione Abrahæ, patris nostri in fide, de Ur Chaldæorum
		      saeculo vigesimo primo;\\
		      ab egressu populi Israël de Ægypto, Moyse duce, saeculo decimo
		      tertio;\\
		      ab unctione David in regem, anno circiter millesimo;\\
		      hebdomada sexagesima quinta, juxta Danielis prophetiam;\\
		      Olympiade centesima nonagesima quarta;\\
		      ab Urbe condita anno septingentesimo quinquagesimo secundo;\\
		      anno imperii Cæsaris Octaviani Augusti quadragesimo secundo;\\
		      toto orbe in pace composito, \textbf{Jesus Christus}, æternus Deus
		      æternique Patris Filius, mundum volens adventu suo piissimo
		      consecrare, de Spiritu Sancto conceptus, novemque post
		      conceptionem decursis mensibus, in Bethlehem Judæ \textbf{nascitur
		      ex Maria Virgine factus homo}:}
	      \smallbreak
	\item[] Tu zapala się światła w kościele.
	      \smallbreak
	      \inde{Nativitas Domini nostri \textbf{Jesu Christi} secundum carnem.}
	      %
	\item Tu bije się we wszystkie dzwonki (podobnie jak w czasie
	      \textit{Gloria} w trakcie Wigilii Paschalnej) i odzywają się organy.
	\item Rozpoczyna się śpiew pieśni Wśród Nocnej Ciszy. Kapłan w tym czasie
	      odkrywa figurkę Dzieciątka, nakłada kadzidło do kadzielnicy i okadza
	      figurkę Dzieciątka. Można też teraz przyozdobić żłóbek kwiatami.
	\item Po zakończeniu śpiewu wszyscy siadają, a kapłan udaje się na ambonę i
	      wygłasza homilię.
	\item Po homilii kapłan ubiera szaty mszalne koloru białego i rozpoczyna
	      Mszę jak zwykle.
\end{itemize}

\section{Zmiany na Mszy Świętej}

\begin{itemize}
	\item Podczas lekcji imię \textbf{Jezus} pada dwa razy -- na oba razy
	      skłaniamy głowę w stronę krzyża
	\item Jeśli wystarczy ministrantów to robimy lucenarium
	\item Jest bożonarodzeniowe \textit{Communicantes}
\end{itemize}


% \footer{
% 	Nihil Obstat	-- Karol Pyziołek 			\hfill 
% 	Imprimatur 		-- ks. Ireneusz Bakalarczyk \hfill 
% 	Skład 			-- Michał Siemaszko}
